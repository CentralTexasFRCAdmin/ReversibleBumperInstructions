%%%%%%%%%%%%%%%%%%%%%%%%%%%%%%%%%%%%%%%%%
% Arsclassica Article
% LaTeX Template
% Version 1.1 (1/8/17)
%
% This template has been downloaded from:
% http://www.LaTeXTemplates.com
%
% Original author:
% Lorenzo Pantieri (http://www.lorenzopantieri.net) with extensive modifications by:
% Vel (vel@latextemplates.com)
%
% Edits: DD/MM/YYYY - Editor (email) - Changes
% 29/12/2018 - Joshua Bryant (jbryant57.2587@gmail.com) - Updated license  to CC BY-SA 4.0
%    Created first version of instructions based on writen instructions from Judy Bryant.
%
% License:
% CC BY-SA 4.0 (https://creativecommons.org/licenses/by-sa/4.0/)
%
%%%%%%%%%%%%%%%%%%%%%%%%%%%%%%%%%%%%%%%%%

%----------------------------------------------------------------------------------------
%	PACKAGES AND OTHER DOCUMENT CONFIGURATIONS
%----------------------------------------------------------------------------------------

\documentclass[
10pt, % Main document font size
a4paper, % Paper type, use 'letterpaper' for US Letter paper
oneside, % One page layout (no page indentation)
%twoside, % Two page layout (page indentation for binding and different headers)
headinclude,footinclude, % Extra spacing for the header and footer
BCOR5mm, % Binding correction
]{scrartcl}

%%%%%%%%%%%%%%%%%%%%%%%%%%%%%%%%%%%%%%%%%
% Arsclassica Article
% Structure Specification File
%
% This file has been downloaded from:
% http://www.LaTeXTemplates.com
%
% Original author:
% Lorenzo Pantieri (http://www.lorenzopantieri.net) with extensive modifications by:
% Vel (vel@latextemplates.com)
%
% Edits: DD/MM/YYYY - Editor (email) - Changes
% 29/12/2018 - Joshua Bryant (jbryant57.2587@gmail.com) - Added tikz image and image scaling
%    structure. Updated license  to CC BY-SA 4.0.
%
% License:
% CC BY-SA 4.0 (https://creativecommons.org/licenses/by-sa/4.0/)
%
%%%%%%%%%%%%%%%%%%%%%%%%%%%%%%%%%%%%%%%%%

%----------------------------------------------------------------------------------------
%	REQUIRED PACKAGES
%----------------------------------------------------------------------------------------

\usepackage[
nochapters, % Turn off chapters since this is an article        
beramono, % Use the Bera Mono font for monospaced text (\texttt)
eulermath,% Use the Euler font for mathematics
pdfspacing, % Makes use of pdftex’ letter spacing capabilities via the microtype package
dottedtoc % Dotted lines leading to the page numbers in the table of contents
]{classicthesis} % The layout is based on the Classic Thesis style

\usepackage{arsclassica} % Modifies the Classic Thesis package

\usepackage[T1]{fontenc} % Use 8-bit encoding that has 256 glyphs

\usepackage[utf8]{inputenc} % Required for including letters with accents

\usepackage{graphicx} % Required for including images
\graphicspath{{Figures/}} % Set the default folder for images
\usepackage{placeins}

\usepackage{enumitem} % Required for manipulating the whitespace between and within lists

\usepackage{lipsum} % Used for inserting dummy 'Lorem ipsum' text into the template

\usepackage{subcaption} % Required for creating figures with multiple parts (subfigures)

\usepackage{amsmath,amssymb,amsthm} % For including math equations, theorems, symbols, etc

\usepackage{varioref} % More descriptive referencing

\usepackage{tabularx}

%----------------------------------------------------------------------------------------
%	tikz image and image scaling stuff
%----------------------------------------------------------------------------------------

\usepackage{tikz}

\usepackage{environ}

\makeatletter
\newsavebox{\measure@tikzpicture}

\NewEnviron{scaletikzpicturetowidth}[1]{%
	\def\tikz@width{#1}%
	\def\tikzscale{1}\begin{lrbox}{\measure@tikzpicture}%
		\BODY
	\end{lrbox}%
	\pgfmathparse{#1/\wd\measure@tikzpicture}%
	\edef\tikzscale{\pgfmathresult}%
	\BODY
}
\makeatother

\usepackage{tikz-dimline}
\tikzset{no slope/.code={\pgfslopedattimefalse}}

%----------------------------------------------------------------------------------------
%	THEOREM STYLES
%---------------------------------------------------------------------------------------

\theoremstyle{definition} % Define theorem styles here based on the definition style (used for definitions and examples)
\newtheorem{definition}{Definition}

\theoremstyle{plain} % Define theorem styles here based on the plain style (used for theorems, lemmas, propositions)
\newtheorem{theorem}{Theorem}

\theoremstyle{remark} % Define theorem styles here based on the remark style (used for remarks and notes)

%----------------------------------------------------------------------------------------
%	HYPERLINKS
%---------------------------------------------------------------------------------------

\hypersetup{
%draft, % Uncomment to remove all links (useful for printing in black and white)
colorlinks=true, breaklinks=true, bookmarks=true,bookmarksnumbered,
urlcolor=webbrown, linkcolor=RoyalBlue, citecolor=webgreen, % Link colors
pdftitle={}, % PDF title
pdfauthor={\textcopyright}, % PDF Author
pdfsubject={}, % PDF Subject
pdfkeywords={}, % PDF Keywords
pdfcreator={pdfLaTeX}, % PDF Creator
pdfproducer={LaTeX with hyperref and ClassicThesis} % PDF producer
} % Include the structure.tex file which specified the document structure and layout

\hyphenation{Fortran hy-phen-ation} % Specify custom hyphenation points in words with dashes where you would like hyphenation to occur, or alternatively, don't put any dashes in a word to stop hyphenation altogether

%----------------------------------------------------------------------------------------
%	TITLE AND AUTHOR(S)
%----------------------------------------------------------------------------------------

\title{\normalfont\spacedallcaps{Reversible Bumper Instructions}} % The article title

%\subtitle{Subtitle} % Uncomment to display a subtitle

\author{\spacedlowsmallcaps{Judy Bryant \& Joshua Bryant}} % The article author(s) - author affiliations need to be specified in the AUTHOR AFFILIATIONS block

\date{\today} % An optional date to appear under the author(s)

%----------------------------------------------------------------------------------------

\begin{document}

%----------------------------------------------------------------------------------------
%	HEADERS
%----------------------------------------------------------------------------------------

\renewcommand{\sectionmark}[1]{\markright{\spacedlowsmallcaps{#1}}} % The header for all pages (oneside) or for even pages (twoside)
%\renewcommand{\subsectionmark}[1]{\markright{\thesubsection~#1}} % Uncomment when using the twoside option - this modifies the header on odd pages
\lehead{\mbox{\llap{\small\thepage\kern1em\color{halfgray} \vline}\color{halfgray}\hspace{0.5em}\rightmark\hfil}} % The header style

\pagestyle{scrheadings} % Enable the headers specified in this block

%----------------------------------------------------------------------------------------
%	TABLE OF CONTENTS & LISTS OF FIGURES AND TABLES
%----------------------------------------------------------------------------------------

\maketitle % Print the title/author/date block

\setcounter{tocdepth}{2} % Set the depth of the table of contents to show sections and subsections only

\tableofcontents % Print the table of contents

\listoffigures % Print the list of figures

%\listoftables % Print the list of tables

%----------------------------------------------------------------------------------------
%	ABSTRACT
%----------------------------------------------------------------------------------------

\section*{Abstract} % This section will not appear in the table of contents due to the star (\section*)

This document will cover how to make straight and corner reversible bumpers according to the 2018 robot rules. By the end of this document, you should have one full set of reversible bumpers ready to mount to the robot. This document is intended for those with at least basic or better skills in sewing and pattern making. This document is not intended to serve as an introduction to sewing.

%----------------------------------------------------------------------------------------

\newpage % Start the article content on the second page, remove this if you have a longer abstract that goes onto the second page

%----------------------------------------------------------------------------------------
%	General Advice/Information
%----------------------------------------------------------------------------------------
\section{General Advice/Information}

This document will cover how to make straight and corner reversible bumpers. When straight bumpers <cut off text>. The wood is not connected at the corners as you would have for continuous bumpers. Corner bumpers are one bumper for each corner of the robot. The wood is connected together and just over the corner and at least 6" out on each side of the bumper. This process does not lend itself well to continuous bumpers and some other process will need to be used if continuous bumpers are needed. In <cut off text>. If there are any conflicts between what you are told in this document in regards to the bumpers and what the rules say, the official rules take precedence.\\

Some general information for reading these instructions:
\begin{itemize}
	\item $\frac{1}{2}"$ seam allowance is used unless otherwise noted.
	\item \textbf{These instructions are based on the bumper rules from 2018. Before making your bumpers, please review the rules for 2019 and make appropriate adjustments for any changes.}
	\item You should work with the chassis build team to determine the type of bumpers needed, the length of each bumper, the mounting points, and mounting hardware placement. it is important to have all this information prior to making your pattern and sewing the bumpers.
	\item The bumpers will have to be removed for inspections. This should be a consideration when planning on the mounting of the bumpers to the chassis. You will not have to remove the bumpers between matches, just at inspection time.
	\item If making straight bumpers, the wood should be cut to the appropriate length, mounting hardware that will be attached to the wood should be attached, and pool noddles should be loosely attached to the wood with duct or masking tape.
	\item Verify ---- bumpers at each step.
	\item Ease is built into the pattern measurements. The ease is important to allow the pool noddle to take the energy of any contact during the competition. If the fabric is attached too tightly, the fabric will take the energy and you will have a higher risk of fabric failure in high velocity impact games. (Save the tight fabric for show bumpers, eased fabric bumpers for competition).
\end{itemize}

%----------------------------------------------------------------------------------------
%	Supplied Needed
%----------------------------------------------------------------------------------------
\section{Supplies Needed}

Supplies needed for making bumpers:
\begin{itemize}
	\item Measuring Tool
	\item Scissors
	\item 4 yards bumper fabric (Duck Canvas is recommended due to it's strength, durability, and flexibility)
		\subitem 2 yards red bumper fabric
		\subitem 2 yards blue bumper fabric
	\item Hook and Loop fastener (sew in version). Blue and Red is recommended but black is acceptable in 2018 robot rules
	\item Thread
	\item 8 sets Iron on Numbers (4 sets for blue bumpers, 4 sets for red bumpers)
	\item Iron
	\item Ironing pad or ironing board with heat resistant cover
	\item Sewing Machine
	\item Machine Needles (you will have better results if you change the needle before beginning the project, after sewing the bumpers and before applying the hook and loop fasteners, and again half way through applying the hook and loop).
	\item Large sheet of paper for making your pattern (bulletin board paper or rolled craft paper are good options)
	\item Pencil
	\item Straight Edge
	\item Pool Noodles
	\item $\frac{3}{4}"$ wood cut to size (if making corner bumpers, wood should be fastened together with wood screws)
	\item Mounting Hardware
	\item Staple Gun and Staples
\end{itemize}

%----------------------------------------------------------------------------------------
%	MAKING THE PATTERN
%----------------------------------------------------------------------------------------

\section{Making the Pattern}

This section contains instructions for both corner bumpers and straight bumpers. The corner bumpers are identical to the straight bumpers with an additional notch cut in them to allow for the corner bend. For both types of bumpers, cut four of each color. If the robot is not square, adjust the measurements labeled A, C, and F to accommodate.

\subsection{Straight Bumper Pattern}

\begin{figure}[h]
	\begin{center}
		\begin{scaletikzpicturetowidth}{\textwidth}
			\begin{tikzpicture}[scale=\tikzscale]
				\draw (0,2.25) -- (4.25,4.25) -- (3.25,0) -- (23.25,0) -- (22.25,4.25) -- (26.5,2.25) -- (26.5,8.75) -- (22.25,8.75) -- (22.25,13) -- (4.25,13) -- (4.25,8.75) -- (0,8.75) -- cycle;
				
				\dimline[line style={thick}]{(4.25,13.7)}{(22.25,13.7)}{A};
				\dimline[line style={thick},label style={above=5mm, no slope}]{(10,0)}{(10,13)}{B};
				\dimline[line style={thick}]{(0,6.25)}{(26.5,6.25)}{C};
				\dimline[line style={thick}]{(22.25,8.1)}{(26.5,8.1)}{D};
				\dimline[line style={thick},label style={no slope}]{(21.6,13)}{(21.6,8.75)}{D};
				\dimline[line style={thick}]{(0,8.1)}{(4.25,8.1)}{D};
				\dimline[line style={thick},label style={no slope}]{(4.9,8.75)}{(4.9,13)}{D};
				\dimline[line style={thick},label style={no slope}]{(-0.6,2.25)}{(-0.6,8.75)}{E};
				\dimline[line style={thick},label style={no slope}]{(27.1,2.25)}{(27.1,8.75)}{E};
				\dimline[line style={thick}]{(3.25,-0.65)}{(23.25,-0.65)}{F};
			\end{tikzpicture}
		\end{scaletikzpicturetowidth}
	\end{center}
	\caption[Straight Bumper Pattern]{Pattern for straight bumpers with labeled lengths to mark.} \label{fig:straightBumperPattern}
\end{figure}
\FloatBarrier

\begin{enumerate}[label=\Alph*]

	\item %Measurement A
	\begin{tabular}{|c|c|}
		\hline
		Item Description & Measurement \\ \hline
		Length of Bumper & $x"$ \\ \hline
		Seam Allowance (x2, one on each end) & $\frac{1}{2}" * 2 = 1"$ \\ \hline
		Total & $(x + 1)"$ \\ \hline
	\end{tabular}

	\item %Measurement B
	\begin{tabular}{|c|c|}
		\hline
		Item Description & Measurement \\ \hline
		Attachment to back of board & $1"$ \\ \hline
		Thickness of wood backing & $\frac{3}{4}"$ \\ \hline
		Height of pool noodle & $2\frac{1}{2}"$ \\ \hline
		Width of pool noodle x2 (height of bumper) & $2\frac{1}{2}" * 2 = 5"$ \\ \hline
		Height of pool noodle & $2\frac{1}{2}"$ \\ \hline
		Thickness of wood backing & $\frac{3}{4}"$ \\ \hline
		Seam Allowance & $\frac{1}{2}"$ \\ \hline
		Total & $13"$ \\ \hline
	\end{tabular}
	
	\item %Measurement C
	\begin{tabular}{|c|c|}
		\hline
		Item Description & Measurement \\ \hline
		Attachment to wood backing & $1"$ \\ \hline
		Thickness of wood backing & $\frac{3}{4}"$ \\ \hline
		Width of pool noodle & $2\frac{1}{2}"$ \\ \hline
		Length of bumper & $x"$ \\ \hline
		Width of pool noodle & $2\frac{1}{2}"$ \\ \hline
		Thickness of wood backing & $\frac{3}{4}"$ \\ \hline
		Attachment to wood backing & $1"$ \\ \hline
		Total & $(x + 8\frac{1}{2})"$ \\ \hline
	\end{tabular}
	
	\item %Measurement D
	\begin{tabular}{|c|c|}
		\hline
		Item Description & Measurement \\ \hline
		Attachment to wood backing & $1"$ \\ \hline
		Thickness of wood backing & $\frac{3}{4}"$ \\ \hline
		Width of pool noodle & $2\frac{1}{2}"$ \\ \hline
		Total & $4\frac{1}{4}"$ \\ \hline
	\end{tabular}
	
	\item %Measurement E
	\begin{tabular}{|c|c|}
		\hline
		Item Description & Measurements \\ \hline
		Seam Allowance & $\frac{1}{2}"$ \\ \hline
		Height of pool noodle x2 (height of bumper) & $2\frac{1}{2}" * 2 = 5"$ \\ \hline
		Easement & $\frac{1}{2}"$ \\ \hline
		Seam Allowance & $\frac{1}{2}"$ \\ \hline
		Total & $6\frac{1}{2}"$\\ \hline
	\end{tabular}
	
	\item %Measurement F
	\begin{tabular}{|c|c|}
		\hline
		Item Description & Measurements \\ \hline
		Seam Allowance & $\frac{1}{2}"$ \\ \hline
		Easement & $\frac{1}{2}"$ \\ \hline
		Length of bumper & $x"$ \\ \hline
		Easement & $\frac{1}{2}"$ \\ \hline
		Seam Allowance & $\frac{1}{2}"$ \\ \hline
		Total & $(x + 2)"$ \\ \hline
	\end{tabular}

\end{enumerate}

\FloatBarrier

\subsection{Corner Bumper Pattern}

\begin{figure}[h]
	\begin{center}
		\begin{scaletikzpicturetowidth}{\textwidth}
			\begin{tikzpicture}[scale=\tikzscale]
				\draw (0,2.25) -- (4.25,3.75) -- (2.75,0) -- (17.25,0) -- (18.75,3.75) -- (20.25,0) -- (34.75,0) -- (33.25,3.75) -- (37.5,2.25) -- (37.5,8.75) -- (33.25,8.75) -- (33.25,13) -- (20.25,13) -- (18.75,8.75) -- (17.25,13) -- (4.25,13) -- (4.25,8.75) -- (0,8.75) -- cycle;
				\dimline[line style={thick}]{(4.25,13.9)}{(17.25,13.9)}{A};
				\dimline[line style={thick}]{(20.25,13.9)}{(33.25,13.9)}{A'};
				\dimline[line style={thick}]{(0,6.25)}{(37.5,6.25)}{C};
				\dimline[line style={thick},label style={above=5mm,no slope}]{(14,0)}{(14,13)}{B};
				\dimline[line style={thick}]{(0,7.9)}{(4.25,7.9)}{D};
				\dimline[line style={thick},label style={no slope}]{(5.15,8.75)}{(5.15,13)}{D};
				\dimline[line style={thick},label style={no slope}]{(32.35,13)}{(32.35,8.75)}{D};
				\dimline[line style={thick}]{(33.25,7.9)}{(37.5,7.9)}{D};
				\dimline[line style={thick},label style={no slope}]{(-0.8,2.25)}{(-0.8,8.75)}{E};
				\dimline[line style={thick},label style={no slope}]{(38.3,2.25)}{(38.3,8.75)}{E};
				\dimline[line style={thick}]{(2.75,-0.9)}{(17.25,-0.9)}{F};
				\dimline[line style={thick}]{(20.25,-0.9)}{(34.75,-0.9)}{F'};
			\end{tikzpicture}
		\end{scaletikzpicturetowidth}
	\end{center}
	\caption[Corner Bumper Pattern]{Pattern for the corner bumpers, including corner notch cutout, with labeled lengths to mark.} \label{fig:cornerBumperPattern}
\end{figure}
\FloatBarrier

\begin{enumerate}[label=\Alph*]

	\item %Measurement A
	\begin{tabular}{|c|c|}
		\hline
		Item Description & Measurement \\ \hline
		Length of Bumper & $x"$ \\ \hline
		Outside Seam Allowance & $\frac{1}{2}"$ \\ \hline
		Corner Seam Allowance & $\frac{1}{2}"$ \\ \hline
		Total & $(x + 1)"$ \\ \hline
	\end{tabular}

	\item %Measurement B
	\begin{tabular}{|c|c|}
		\hline
		Item Description & Measurement \\ \hline
		Attachment to wood backing & $1"$ \\ \hline
		Thickness of wood backing & $\frac{3}{4}"$ \\ \hline
		Height of pool noodle & $2\frac{1}{2}"$ \\ \hline
		Height of bumper (2x width of pool noodle) & $2\frac{1}{2}" * 2 = 5"$ \\ \hline
		Height of pool noodle & $2\frac{1}{2}"$ \\ \hline
		Thickness of wood backing & $\frac{3}{4}"$ \\ \hline
		Seam Allowance & $\frac{1}{2}"$\\ \hline
		Total & $13"$ \\ \hline
	\end{tabular}
	
	\item %Measurement C
	\begin{tabular}{|c|c|}
		\hline
		Item Description & Measurement \\ \hline
		Attachment to wood backing & $1"$ \\ \hline
		Thickness of wood backing & $\frac{3}{4}"$ \\ \hline
		Width of pool noodle & $2\frac{1}{2}"$ \\ \hline
		Length of bumper & $x"$ \\ \hline
		Width of pool noodle (*2 for noodle overlap in corner) & $2\frac{1}{2}" * 2 = 5"$ \\ \hline
		Length of bumper & $x'"$ \\ \hline
		Width of pool noodle & $2\frac{1}{2}"$ \\ \hline
		Thickness of wood backing & $\frac{3}{4}"$ \\ \hline
		Attachment to wood backing & $1"$ \\ \hline
		Total & $(x + x' + 13.5)"$ \\ \hline
	\end{tabular}
	$x$ and $x'$ are only used to denote if one side of the bumpers is longer than the other. They correspond to the A, A', F, and F' marked on Figure  \ref{fig:cornerBumperPattern}. If both sides are the same, then this reduces to $(2*x + 13.5)"$ for the total length.

	\item %Measurement D
	\begin{tabular}{|c|c|}
		\hline
		Item Description & Measurement \\ \hline
		Attachment to wood backing & $1"$ \\ \hline
		Thickness of wood backing & $\frac{3}{4}"$ \\ \hline
		Width of pool noodle & $2\frac{1}{2}"$ \\ \hline
		Total & $4\frac{1}{4}"$ \\ \hline
	\end{tabular}
	
	\item %Measurement E
	\begin{tabular}{|c|c|}
		\hline
		Item Description & Measurement \\ \hline
		Seam Allowance & $\frac{1}{2}"$ \\ \hline
		Height of Bumper (*2 height of pool noodle) & $2\frac{1}{2}" * 2 = 5"$ \\ \hline
		Easement & $\frac{1}{2}"$ \\ \hline
		Seam Allowance & $\frac{1}{2}"$ \\ \hline
		Total & $6\frac{1}{2}"$ \\ \hline
	\end{tabular}
	
	\item %Measurement F
	\begin{tabular}{|c|c|}
		\hline
		Item Description & Measurement \\ \hline
		Seam Allowance & $\frac{1}{2}"$ \\ \hline
		Easement & $\frac{1}{2}"$ \\ \hline
		Length of bumper & $x"$ \\ \hline
		Easement & $\frac{1}{2}"$ \\ \hline
		Seam Allowance & $\frac{1}{2}"$ \\ \hline
		Total & $(x + 2)"$ \\ \hline
	\end{tabular}
\end{enumerate}

%----------------------------------------------------------------------------------------
%	CUTTING THE FABRIC
%----------------------------------------------------------------------------------------

\section{Cutting the Fabric}

Once you have a pattern, you are ready to begin work with fabric. for straight bumpers, you will cut 4 of each color. For corner bumpers that are the minimum 6" on each side of the corner, you will cut 4 of each color. If one side of the bumpers will be longer than the other, you will cut 2 of each color with the pattern face up and 2 of each folor with the pattern face down (for experienced sewers, cut 2 of each color with right side of fabric together).

Now take a moment to decide which color will be attached to the top of the bumper and which color will be attached at the bottom. It might help to mark the fabric within the 1" attachment allowance with a top or bottom designation. This will help when you are applying the team numbers and securing the fabric.

At this point, I like to do a simple zig zag finish around each piece of fabric. It is not necessary, but I like the neatness that this step brings to the finished product.

%----------------------------------------------------------------------------------------
%	TEAM NUMBERS
%----------------------------------------------------------------------------------------

\section{Team Numbers}

The team numbers are required to be displayed on all four sides of the bumpers. The numbers must be at least 4" tall with at least 1/2" stroke. Most teams use one of the 3 following methods for team numbers:
\begin{itemize}
	\item Iron on numbers
	\item Make a template and paint on the fabric
	\item Make appliques and sew the numbers on.
\end{itemize}
If you are making corner bumpers and have a 4-digit team number, you will probably need to split the numbers and place 2 digits on one side and the other 2 digits on the other side. Plan this placement carefully in order to have the numbers correctly placed on the bumpers. For example with team number 1234:

\begin{figure}[h]
	\centering
	\begin{scaletikzpicturetowidth}{\textwidth}
		\begin{tikzpicture}[scale=\tikzscale]
			\draw (6,3) -- (6,4) -- (5,4);
			\draw (2,4) -- (1,4) -- (1,3);
			\draw (1,2) -- (1,1) -- (2,1);
			\draw (5,1) -- (6,1) -- (6,2);
			
			\node at (5.5,3.5) {\large\rotatebox{315}{Corner A}};
			\node at (6.3,3.5) {\large\rotatebox{90}{34}};
			\node at (5.5,4.3) {\large\rotatebox{180}{12}};
			
			\node at (1.5,3.5) {\large\rotatebox{45}{Corner B}};
			\node at (1.5,4.3) {\large\rotatebox{180}{34}};
			\node at (0.7,3.5) {\large\rotatebox{270}{12}};
		
			\node at (1.5,1.5) {\large\rotatebox{315}{Corner C}};
			\node at (0.7,1.5) {\large\rotatebox{270}{34}};
			\node at (1.5,0.7) {\large12};
			
			\node at (5.5,1.5) {\large\rotatebox{45}{Corner D}};
			\node at (5.5,0.7) {\large34};
			\node at (6.3,1.5) {\large\rotatebox{90}{12}};
		\end{tikzpicture}
	\end{scaletikzpicturetowidth}
	\caption[Bumper Number Allignment, Top View]{Allignment of team numbers on bumpers so that each side of the bumpers display the full team number across the gap between bumpers.} \label{fig:topBumperNumbers}
\end{figure}
	
	
\begin{figure}[h]
	\begin{scaletikzpicturetowidth}{\textwidth}
		\begin{tikzpicture}[scale=\tikzscale]
			\filldraw[fill=red, draw=red] (0,2.25) -- (4.25,3.75) -- (2.75,0) -- (17.25,0) -- (18.75,3.75) -- (20.25,0) -- (34.75,0) -- (33.25,3.75) -- (37.5,2.25) -- (37.5,8.75) -- (33.25,8.75) -- (33.25,13) -- (20.25,13) -- (18.75,8.75) -- (17.25,13) -- (4.25,13) -- (4.25,8.75) -- (0,8.75) -- cycle;
			\draw[dashed, ultra thick] (4.25,3.75) rectangle (17.25,8.75);
			\draw[dashed, ultra thick] (20.25,3.75) rectangle (33.25,8.75);
			\node[text=white] at (10.75,6.25) {\Huge\textbf{$34$}};
			\node[text=white] at (26.75,6.25) {\Huge\textbf{$12$}};
			\dimline[line style={thick}]{(4.25,13.9)}{(17.25,13.9)}{Bumper Attachment Edge};
			\dimline[line style={thick}]{(20.25,13.9)}{(33.25,13.9)}{Bumper Attachment Edge};
				
			\filldraw[fill=blue, draw=blue] (0,-3.25) -- (4.25,-4.75) -- (2.75,-1) -- (17.25,-1) -- (18.75,-4.75) -- (20.25,-1) -- (34.75,-1) -- (33.25,-4.75) -- (37.5,-3.25) -- (37.5,-9.75) -- (33.25,-9.75) -- (33.25,-14) -- (20.25,-14) -- (18.75,-9.75) -- (17.25,-14) -- (4.25,-14) -- (4.25,-9.75) -- (0,-9.75) -- cycle;
			\draw[dashed, ultra thick] (4.25,-4.75) rectangle (17.25,-9.75);
			\draw[dashed, ultra thick] (20.25,-4.75) rectangle (33.25,-9.75);
			\node[text=white] at (10.75,-7.25) {\Huge\textbf{$34$}};
			\node[text=white] at (26.75,-7.25) {\Huge\textbf{$12$}};
			\dimline[line style={thick}]{(4.25,-14.9)}{(17.25,-14.9)}{Bumper Attachment Edge};
			\dimline[line style={thick}]{(20.25,-14.9)}{(33.25,-14.9)}{Bumper Attachment Edge};
		\end{tikzpicture}
	\end{scaletikzpicturetowidth}
	\caption[Bumper Number Allignment, Side View]{Allignment of Team Numbers on bumpers with red bumper fabric attached to bumper backing on top and blue bumper fabric mounted at bottom of bumper backing. The dashed outlines show the region team numbers should appear in.} \label{fig:sideBumperNumbers}
\end{figure}
\FloatBarrier

For straight bumpers, measure to the center and place the numbers in order (1234). You do have to make sure that you know which color will be attached at the top and which color attached at the bottom.

%----------------------------------------------------------------------------------------
%	SEWING THE BUMPERS
%----------------------------------------------------------------------------------------

\section{Sewing the Bumpers}

Note: Right side of fabric refers to the side of the fabric with the team numbers. All seams are made with right sides together unless noted otherwise.

For straight and corner bumpers, start with the red bumper fabric for a bumper:
\begin{enumerate}
	\item On each end of the fabric, sew D and D together. For corner bumpers between the 2 A's also sew the "V" together (forms the corner). \label{steps:corners}
	\item On the flap side, sew the end notches together. For the corner bumpers, leave the center notch unsewn for now. \label{steps:flaps}
	\item Repeat steps \ref{steps:corners} and \ref{steps:flaps} for blue fabric for the same bumper.
	\item Mark the center of the E section at the end of each bumper.
	\item Place right sides together for the red and blue fabric of the same bumper matching center marking and end notch seams on flap side of the bumper.
	\item Sew flap side of bumper from center marking to center marking using a 1" seam allowance from center marking to notch seam, 1/2" allowance from notch seam to notch seam, and 1" from notch seam to center marking. For corner bumpers, sew the notch in the center (corner) coming to a point near the point of the notch. Reinforce the point y sewing over the seam. Trim the fabric at the point to reduce the bulk when turning. This creates a flap that will overlap at the corner and allowing enough to give to quickly turn the flap reversing the color showing on the robot.
	\item Clip the fabric at center mark and make a reinforcing stitch.
	\item Turn the bumpers resulting in the right side of the fabric facing out.
	\item Test fit of bumpers on attachment sides to the wood over the noodles.
	\item Make sure team number will display roperly on both colors and flap will turn properly.
	\item Top stitch close to folded flap edges from center mark to center mark.
	\item Mark center line from side E to side E on both colors of the bumper.
	\item Pin both colors together matching center line.
	\item Sew along center line. This forms the flap that will be turned from top to bottom or bottom to top changing the color of the bumper on the robot.
	\item Perform another test fit on your bumper. Make sure the flap will turn properly.
	\item Verify that you have 1" of fabric that will fold under the wood for attaching the fabric to the wood. This fabric will end up between the wood and the chassis frame of the robot.
	\item Mark the edge of the wood on the fabric. This is important because the hook and loop fastener has to be supported by the wood per the robot rules.
	\item Sew the hook portion of the hook and loop to the attachment side of the fabric (on the right side of fabric). \label{steps:hook}
	\subitem If using blue and red hook and loop, sew the blue on the blue fabric and the red on the red fabric.
	\item Sew the loop portion of the hook and loop to the flap of the fabric (F). \label{steps:loop}
	\item Repeat steps \ref{steps:hook} and \ref{steps:loop} for the other color fabric. On the second color for the flap, you will be sewing through 2 layers  of Loop.
	\subitem I recommend changing your needle between each bumper. Otherwise, you will probably have thread fraying issues.
	\item Verify proper fit to bumpers and verify flaps will turn properly.
	\item For corner bumpers, trim excess fabric from corner.
	\item Trim fabric around mounting hardware.
	\item Making sure that hook and loop is supported by the 3/4" edge of the wood. Staple fabric to the back of the bumper wood (side of the wood that will be facing the chassis).
	\item Repeat process for remaining bumper sections.
\end{enumerate}

%----------------------------------------------------------------------------------------
%	BIBLIOGRAPHY
%----------------------------------------------------------------------------------------

%----------------------------------------------------------------------------------------

\end{document}
